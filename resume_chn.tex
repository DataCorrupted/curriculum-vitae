%%%%%%%%%%%%%%%%%%%%%%%%%%%%%%%%%%%%%%%%%
% Medium Length Professional CV
% LaTeX Template
% Version 2.0 (8/5/13)
%
% This template has been downloaded from:
% http://www.LaTeXTemplates.com
%
% Original author:
% Trey Hunner (http://www.treyhunner.com/)
%
% Important note:
% This template requires the resume.cls file to be in the same directory as the
% .tex file. The resume.cls file provides the resume style used for structuring the
% document.
%
%%%%%%%%%%%%%%%%%%%%%%%%%%%%%%%%%%%%%%%%%

%----------------------------------------------------------------------------------------
%   PACKAGES AND OTHER DOCUMENT CONFIGURATIONS
%----------------------------------------------------------------------------------------
\documentclass{resume} % Use the custom resume.cls style
\usepackage[left=0.75in,top=0.3in,right=0.75in,bottom=0.3in]{geometry} % Document margins

\name{荣宇阳} % Your name
\address{中国上海,华夏中路393号,201201} % Your address
% \address{123 Pleasant Lane \\ City, State 12345} % Your secondary addess (optional)
\address{(+86)177~$\cdot$~2105~$\cdot$~4096 \\ PeterRong96@gmail.com} % Your phone number and email

\begin{document}

%----------------------------------------------------------------------------------------
%   EDUCATION SECTION
%----------------------------------------------------------------------------------------

\begin{rSection}{Education}

\begin{rSubsection}
  {上海科技大学}{2015.8 - 现在}{计算机科学}{上海, 中国}
    \item 总GPA 3.79, 专业GPA 3.89.
\end{rSubsection}

%------------------------------------------------

\begin{rSubsection}
  {芝加哥大学}{2016.7 - 2016.8}{任选课程}{芝加哥,伊利诺伊州}
    \item 强化英语及数值分析,成绩均为A。
\end{rSubsection}

\end{rSection}

%----------------------------------------------------------------------------------------
%   EXPERIENCE
%----------------------------------------------------------------------------------------
\begin{rSection}{Experience}

\begin{rSubsection}
  {今日头条, 字节跳动}{2018.9 – 现在}{安全实验室实习生}{北京, 中国}
    \item 在Hao Chen教授的指导下在头条安全实验室工作。
    \item 主要工作是软件验证与fuzzing算法。
\end{rSubsection}

%------------------------------------------------

\begin{rSubsection}
  {CSST, 加州大学洛杉矶分校}{2018.7 - 2018.9}{暑研实习生}{洛杉矶, 加利福尼亚州}
    \item 受Lixia Zhang教授指导,研究命名数据网络(Named Data Networking)相关应用
    \item 重心在OpenmHealth. 通过使用NDNCERT重写身份管理模块来升级现有构架。
\end{rSubsection}

%------------------------------------------------

\begin{rSubsection}
  {Screen++}{Jun 2017}{队长}{上海, 中国}
    \item Proposed an application to connect all the screens in different platforms.
    \item Development \& market model. Did final presentation.
    \item Python \& Apache used. Group of 5. Won 3rd prize in iLab Hackathon.
\end{rSubsection}

%------------------------------------------------

\begin{rSubsection}
  {ABB Group}{Oct 2017 – Jul 2018}{Research Intern}{上海, 中国}
    \item Group of 10. Tried to combine robot arm and an indoor mobile car.
    \item Responsible for software development, including \textbf{navigation system} \& \textbf{state machine}.
    \item Python, C++ and ROS used.
\end{rSubsection}

\end{rSection}

%----------------------------------------------------------------------------------------
%   AWARDS
%----------------------------------------------------------------------------------------
\begin{rSection}{Awards}
    \rItem{Outstanding Student}{Nov 2017}{}{}
    \rItem{上海Tech President's Scholarship}{Nov 2016}{}{}
    \rItem{上海 Scholarship}{Oct 2016}{}{}
\end{rSection}

%----------------------------------------------------------------------------------------
%   TECHNICAL STRENGTHS SECTION
%----------------------------------------------------------------------------------------

\begin{rSection}{Technical Strengths}

\begin{tabular}{ @{} >{\bfseries}l @{\hspace{6ex}} l }
Languages & Chinese, English \\
Computer Languages & Rust, Java, C++, C, Python, MATLAB\\
Protocols \& APIs & Android, boost, ROS, numpy \\
\end{tabular}

\end{rSection}

\end{document}
