%%%%%%%%%%%%%%%%%%%%%%%%%%%%%%%%%%%%%%%%%
% Medium Length Professional CV
% LaTeX Template
% Version 2.0 (8/5/13)
%
% This template has been downloaded from:
% http://www.LaTeXTemplates.com
%
% Original author:
% Trey Hunner (http://www.treyhunner.com/)
%
% Important note:
% This template requires the resume.cls file to be in the same directory as the
% .tex file. The resume.cls file provides the resume style used for structuring the
% document.
%
%%%%%%%%%%%%%%%%%%%%%%%%%%%%%%%%%%%%%%%%%

%----------------------------------------------------------------------------------------
%   PACKAGES AND OTHER DOCUMENT CONFIGURATIONS
%----------------------------------------------------------------------------------------
\documentclass{resume} % Use the custom resume.cls style
\usepackage[left=0.75in,top=0.25in,right=0.75in,bottom=0.2in]{geometry} % Document margins

\name{荣宇阳} % Your name
\address{中国上海,华夏中路393号,201201} % Your address
% \address{123 Pleasant Lane \\ City, State 12345} % Your secondary addess (optional)
\address{(+86)177~$\cdot$~2105~$\cdot$~4096 \\ PeterRong96@gmail.com \\ \href{PeterRong.netlify.com}{PeterRong.netlify.com}} % Your phone number and email

\begin{document}

%----------------------------------------------------------------------------------------
%   EDUCATION SECTION
%----------------------------------------------------------------------------------------

\begin{rSection}{教育经历}

\begin{rSubsection}
  {上海科技大学}{2015.8 - 现在}{计算机科学}{中国, 上海}
    \item 总GPA 3.79/4, \textbf{专业GPA 3.89/4}, 排名\textbf{前3\%}.
\end{rSubsection}

%------------------------------------------------

\begin{rSubsection}
  {芝加哥大学}{2016.7 - 2016.8}{任选课程}{伊利诺伊,芝加哥}
    \item 强化英语及数值分析,成绩均为A。
\end{rSubsection}

\end{rSection}

%----------------------------------------------------------------------------------------
%   EXPERIENCE
%----------------------------------------------------------------------------------------
\begin{rSection}{实习经历}

\begin{rSubsection}
  {今日头条, 字节跳动}{2018.9 – 现在}{安全实验室实习生}{中国,北京}
    \item 在Hao Chen教授的指导下在头条安全实验室工作, 主要工作是软件验证与fuzzing算法。
    \item 我的工作目前集中在有约束的情况下出发整数溢出漏洞。我们使用\textbf{LLVM}来给修改源码。
    \item 项目中使用\textbf{Rust, C与C++}。
\end{rSubsection}

%------------------------------------------------

\begin{rSubsection}
  {CSST, 加州大学洛杉矶分校}{2018.7 - 2018.9}{暑研实习生}{加利福尼亚,洛杉矶}
    \item 受Lixia Zhang教授指导,研究\textbf{命名数据网络(Named Data Networking)}相关应用;
    \item 重心在OpenmHealth。 通过使用NDNCERT重写身份管理模块来升级现有构架。
\end{rSubsection}

%------------------------------------------------

\begin{rSubsection}
  {ABB集团}{2017.10 – 2018.7}{研发实习生}{中国,上海}
    \item 十人团队,致力于整合ABB产品Yumi与室内移动机器人;
    \item 负责软件开发,包括\textbf{导航系统}、\textbf{建图系统}与\textbf{状态机};
    \item 使用Python, C++ 和 ROS。
\end{rSubsection}

%------------------------------------------------

\begin{rSubsection}
  {Screen++}{2017.6}{队长}{中国,上海}
    \item 提出了在不同平台与设备之间分享屏幕的想法;
    \item 实际参与开发与市场模型的构建,并完成了最终展示;
    \item 使用Python \& Apache.。五人团队并最终获得\textbf{三等奖}。
\end{rSubsection}

\end{rSection}

%----------------------------------------------------------------------------------------
%   AWARDS
%----------------------------------------------------------------------------------------
\begin{rSection}{获奖情况}
    \rItem{三好学生}{2017.11}{}{}
    \rItem{上海科技大学校长奖学金}{2016.11}{}{}
    \rItem{上海奖学金}{2016.10}{}{}
\end{rSection}

%----------------------------------------------------------------------------------------
%   TECHNICAL STRENGTHS SECTION
%----------------------------------------------------------------------------------------

\begin{rSection}{技能}

\begin{tabular}{ @{} >{\bfseries}l @{\hspace{6ex}} l }
语言 & 中文, 英语(熟练使用) \\
编程语言 & Rust, Java, C++, C, Python, MATLAB\\
工具与接口 & LLVM, Android, boost, ROS, numpy \\
\end{tabular}

\end{rSection}

\end{document}
